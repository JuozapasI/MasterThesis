The single cell RNA sequencing becomes more and more popular tool for analysis of cellular systems.
This technology enables to sequence thousands of cells in a single experiment and provides huge amount of data.
The typicall workflow of the anaysis of such data is to map reads to the known transcriptome and and construct cell-gene matrices,
which are used in the downstream analysis.
However, there are several problems regarding mapping to the known transcriptome:

\begin{enumerate}
  \item The transcriptomes used are not fully comprehensive.
  Even though there are given great efforts to annotate all genes, it is very likelly that not all genes are annotated,
  and many remains to be found.
  Such yet undefined genes are missed in the typical scRNAseq analysis.
  \item The human (and many other species) transcriptome is very complex, with many overlapping features.
  This prevents mapping algorithms to assign short reads to a single feature, usually resulting in discarding such reads from analysis.
\end{enumerate}

Addressing such problems, particularly focusing on mapped, but unassigned reads, could reveal some biologically significant information,
that is currently disregarded in the most current scRNAseq data analysis.
