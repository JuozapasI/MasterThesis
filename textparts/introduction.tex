Single-cell RNA sequencing (scRNA-seq) is becoming an increasingly popular tool for analyzing cellular systems.
This technology enables whole transcriptome sequencing of thousands of cells in a single experiment, generating a vast amount of data.
The typical workflow for analyzing such data involves mapping reads to a known genome reference and constructing cell-gene matrices,
i.e. matrices in which rows correspond to cells, columns correspond to genes,
and each entry indicates the expression level of a gene in a specific cell.
Such matrices are used in downstream analyses.
However, some reads are always mapped to the genome but remain unassigned to any known gene.
Hence, such reads are not used in downstream analysis.

Since these unassigned reads can constitute a significant fraction of the total reads (often up to 30\%),
understanding the reasons behind their origin could help improve scRNA-seq technologies and data analysis.

There are two potential sources of unassigned reads: either they are sequencing artifacts,
or they originate from real transcripts that remain unassigned due to issues related to transcriptomic references.

The two main challenges with transcriptomic references are:

\begin{enumerate}
  \item Incomplete transcriptome annotations –
  even though there are given great efforts to annotate all genes, it is very likelly that not all genes are annotated,
  and many remains to be found.
  Such yet undefined genes are missed in the typical scRNAseq analysis.
  \item Complexity and overlaps in transcriptomic features – the human (and many other species) genome is very complex,
   with many overlapping features.
   In such cases, mapping algorithms assign multiple features for a given read, and such reads are usually discarded from the downstream analysis.
\end{enumerate}

This project focuses on unassigned reads in scRNA-seq datasets, aiming to uncover their origins and 
enhance scRNA-seq data analysis by incorporating biologically meaningful data that is typically disregarded.
